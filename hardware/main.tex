%%%%%%%%%%%%%%%%%%%%%%%%%%%%%%%%%%%%%%%%%
% The Legrand Orange Book
% LaTeX Template
% Version 1.3 (21/8/13)
%
% This template has been downloaded from:
% http://www.LaTeXTemplates.com
%
% Original author:
% Mathias Legrand (legrand.mathias@gmail.com)
%
% License:
% CC BY-NC-SA 3.0 (http://creativecommons.org/licenses/by-nc-sa/3.0/)
%
% Compiling this template:
% This template uses biber for its bibliography and makeindex for its index.
% When you first open the template, compile it from the command line with the 
% commands below to make sure your LaTeX distribution is configured correctly:
%
% 1) pdflatex main
% 2) makeindex main.idx -s StyleInd.ist
% 3) biber main
% 4) pdflatex main x 2
%
% After this, when you wish to update the bibliography/index use the appropriate
% command above and make sure to compile with pdflatex several times 
% afterwards to propagate your changes to the document.
%
% This template also uses a number of packages which may need to be
% updated to the newest versions for the template to compile. It is strongly
% recommended you update your LaTeX distribution if you have any
% compilation errors.
%
% Important note:
% Chapter heading images should have a 2:1 width:height ratio,
% e.g. 920px width and 460px height.
%
%%%%%%%%%%%%%%%%%%%%%%%%%%%%%%%%%%%%%%%%%

%----------------------------------------------------------------------------------------
%	PACKAGES AND OTHER DOCUMENT CONFIGURATIONS
%----------------------------------------------------------------------------------------

\documentclass[11pt,fleqn]{book} % Default font size and left-justified equations

\usepackage[top=3cm,bottom=3cm,left=3.2cm,right=3.2cm,headsep=10pt,a4paper]{geometry} % Page margins

\usepackage{xcolor} % Required for specifying colors by name
\definecolor{ocre}{RGB}{243,102,25} % Define the orange color used for highlighting throughout the book

\newcommand{\degree}{\ensuremath{^\circ}}

% Font Settings
\usepackage{avant} % Use the Avantgarde font for headings
%\usepackage{times} % Use the Times font for headings
\usepackage{mathptmx} % Use the Adobe Times Roman as the default text font together with math symbols from the Sym­bol, Chancery and Com­puter Modern fonts

\usepackage{microtype} % Slightly tweak font spacing for aesthetics
\usepackage[utf8]{inputenc} % Required for including letters with accents
\usepackage[T1]{fontenc} % Use 8-bit encoding that has 256 glyphs

% Bibliography
\usepackage[style=alphabetic,sorting=nyt,sortcites=true,autopunct=true,babel=hyphen,hyperref=true,abbreviate=false,backref=true,backend=biber]{biblatex}
\addbibresource{bibliography.bib} % BibTeX bibliography file
\defbibheading{bibempty}{}

% Index
\usepackage{calc} % For simpler calculation - used for spacing the index letter headings correctly
\usepackage{makeidx} % Required to make an index
\makeindex % Tells LaTeX to create the files required for indexing

%----------------------------------------------------------------------------------------

\input{structure} % Insert the commands.tex file which contains the majority of the structure behind the template

\begin{document}

%----------------------------------------------------------------------------------------
%	TITLE PAGE
%----------------------------------------------------------------------------------------

\begingroup
\thispagestyle{empty}
\AddToShipoutPicture*{\put(6,5){\includegraphics[scale=1]{background}}} % Image background
\centering
\vspace*{9cm}
\par\normalfont\fontsize{35}{35}\sffamily\selectfont
\Huge Brew Master 9000\\
Hardware\\
\textit{Research and reference}\par % Book title
\vspace*{1cm}
{\Large Iver Egge\\
Kristoffer Dalby\\
Johan Slettvold\\
Christer Lund\\} %\par % Author name
\endgroup

%----------------------------------------------------------------------------------------
%	COPYRIGHT PAGE
%----------------------------------------------------------------------------------------

\newpage
~\vfill
\thispagestyle{empty}

\noindent Copyright \copyright\ 2013 Iver Egge\\ % Copyright notice

\noindent \textsc{Published by null}\\ % Publisher

\noindent \textsc{wb3.no}\\ % URL

\noindent Who needs a license?\\ % License information

\noindent The U.S. customary units should have been aborted\\

\noindent \textit{First printing, October 2013} % Printing/edition date

%----------------------------------------------------------------------------------------
%	TABLE OF CONTENTS
%----------------------------------------------------------------------------------------

\chapterimage{chapter_head_1.pdf} % Table of contents heading image

\pagestyle{empty} % No headers

\tableofcontents % Print the table of contents itself

\cleardoublepage % Forces the first chapter to start on an odd page so it's on the right

\pagestyle{fancy} % Print headers again

%----------------------------------------------------------------------------------------
%	CHAPTER 1
%----------------------------------------------------------------------------------------

\chapterimage{tubes_fittings.pdf} % Chapter heading image

\chapter{Tubes and fittings}

\section{Tri Clover / Tri Clamp / Sanitary}\index{Tri Clover / Tri Clamp / Sanitary}

These fittings are used throughout the brewing system as they provide easy dissassembling and just look really, really cool.\\

They consist of two flanges and a gasket that are compressed together with a clamp. The clamp is tightened with a thumb screw.

\subsection{Gaskets}\index{Tri Clover / Tri Clamp / Sanitary!Gaskets}

There are many types of gaskets produced for use with tri clamp fittings. Three of these types are commonly used in amateur brewing systems; silocone, EPDM (ethylene propylene diene monomer) rubber and PTFE (polytetrafluoroethylene) teflon.

\subsubsection{Silicone}

Temperature rating of $-49\degree$C \textasciitilde $230\degree$C.\\
These are the most common gaskets that are available. They will degrade over time if used with strong acids. As they stick very well to metal and are soft, they provide an excellent seal.

\subsubsection{EPDM rubber}

Temperature rating of $-34\degree$C \textasciitilde $149\degree$C.\\
They have better chemical resistance than silicone gaskets and will last longer than silicone gaskets. As the name implies they are made of rubber and are therefor soft and somewhat sticky.

\subsubsection{PTFE teflon}

Temperature rating of $-73\degree$C \textasciitilde $260\degree$C.\\
They have the best chemical resistance of all gaskets and will last the longest. They are, however, hard and will need considerably more compression to provide a good seal.

\subsubsection{Gaskets with flanges}

A normal gasket will fall right of the fitting when loose. If the gasket has a flange that covers the outer part of the fitting it will stay on the fitting when dissassembling (i.e. does not fall into the warm wort).

\subsubsection{Stiffness}

A stiff gasket that is not sticky will allow you to turn the fitting without dissassembling the entire connection. Stiffer gaskets will need more compression to provide a good seal.

\section{Compression fittings}\index{Compression fittings}

Compression fittings consists of a compression nut and ring that slides over the tube and a threaded fitting. If the tube is made of soft metal there should also be a support insert that is inserted into the tube to prevent it from collapsing.\\
These fittings should not be over tightened as this will ruin the compression ring and therefore the seal.

\section{Valves}\index{Valves}

Any normal ball valve will work in a brewing system. Only full port ball valves have the same size hole in the ball as the pipeline.

It should be possible to dissassemble the valves for cleaning and repair. Therefore 3-part valves are recommended for a brewing system.

%----------------------------------------------------------------------------------------
%	CHAPTER 2
%----------------------------------------------------------------------------------------

\chapterimage{pumps.pdf} % Chapter heading image

\chapter{Pump}

\section{Requirements}\index{Pump!Requirements}

A suitable pump for a brewing system have the following requirements

\begin{itemize}
\item All parts have to be of food grade.
\item It should be magnetic coupled so that in the event of the impeller becoming stuck due to malt particles, the motor will not burn out.
\item The lift limit should be at least 2 meters.
\item The temperature rating should be at least that of boiling wort ($100\degree$C).
\item It should be self-priming.
\end{itemize}

\section{Priming}\index{Pump!Priming}

Priming is to fill the pump head with the liquid that is to be pumped. As the pump is stored dry, the contents of the pump head is only air. As air and water have very different physical properties, a pump that is designed to pump a liquid will perform terrible at pumping air and will eventually break.\\

A self-priming pump differs from a non-priming pump that it can pump a mixture of air and liquid. This means that the pump will be able to remove air trapped in the head as long as it has a source of liquid.\\

A pump made to pump liquid should NEVER be run dry.

\section{Mounting}\index{Pump!Mounting}

The pump should be mounted so that liquid enters at the bottom of the head and exits at the top. This is to prevent air pockets from occupying the head. If the pump has to be mounted in a vertical position the head should be placed at the top, not the bottom.\\

Many pumps require that there is pressure at the inlet. Therefore it should be mounted as far as possible below the source of liquid.

\section{Cavitation}

Cavitation is the formation of vapor cavities in a liquid due to pressure differences around the impeller as it turns quickly and with great force. It can be harmful to the pump and will produce a rattle-like sound.

%----------------------------------------------------------------------------------------

\chapterimage{temp.pdf} % Chapter heading image

\chapter{Temperature measurements}

\section{Probes}\index{Probes}

There are many different temperature probes available that either communicates digitally or outputs an analog signal that can be processed for use with various microcontrollers. The most common types seem to be

\begin{itemize}
\item DS18B20 (digital, Dallas 1-wire)
\item LM35 (analog, 0-1V)
\item Thermocouple (analog, needs amplifier)
\end{itemize}

The DS18B20 and LM35 are quite easy to use but lacks good casings for wet use. Thermocouples are the industry standard for temperature measurements and comes in a great number of different casings. They consist of two different types of conductors that provide a voltage difference according to the temperature of the system. This voltage difference is in the order of microvolts and an amplifier is needed to read the output.

\subsection{K-type thermocouples}

This seems to be the most common of the thermocouples types. It consists of chromel-alumel conductors. The temperature range of the probe can be as wide as $-200\degree$C \textasciitilde $1350\degree$C.

The MAX31855 is an amplifier produced by Maxim Integrated. It outputs an analog signal and requires an ADC for use with a microcontroller (or a microcontroller with an integrated ADC). There are libraries available for Ardunio and Raspberry Pi. 

%------------------------------------------------

\section{Placement}

Liquid that is being heated does not have uniformly distributed temperature. There should be two probes per tank, and the average of the two will be the actual temperature of the system.
If only one probe is used to measure the temperature of a tank the probe should be fitted at the middle (why?).

%----------------------------------------------------------------------------------------

\chapterimage{rims.pdf} % Chapter heading image

\chapter{RIMS-tube}

\section{What is RIMS}

RIMS stands for Recirculating Infusion Mash System.\\
The wort is recirculated continously through the malt. The RIMS-tube contains a heating element and a temperature probe that enables the brewer to do step mashing by pumping the wort through the tube. RIMS also provides crystal clear wort as it is filtered continously through the malt.

\section{Components}

A typical RIMS-tube consists of the following

\begin{itemize}
\item Two tee's
\item A heating element adapter and heating element
\item A temperature probe adapater and temperature probe
\item Two hose/pipe connections
\end{itemize}


%----------------------------------------------------------------------------------------
%	BIBLIOGRAPHY
%----------------------------------------------------------------------------------------

\chapter*{Bibliography}
\addcontentsline{toc}{chapter}{\textcolor{ocre}{Bibliography}}
\section*{Books}
\addcontentsline{toc}{section}{Books}
\printbibliography[heading=bibempty,type=book]
\section*{Articles}
\addcontentsline{toc}{section}{Articles}
\printbibliography[heading=bibempty,type=article]

%----------------------------------------------------------------------------------------
%	INDEX
%----------------------------------------------------------------------------------------

\cleardoublepage
\setlength{\columnsep}{0.75cm}
\addcontentsline{toc}{chapter}{\textcolor{ocre}{Index}}
\printindex

%----------------------------------------------------------------------------------------

\end{document}